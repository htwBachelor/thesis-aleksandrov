\chapter{Einleitung}

Seit mehreren Jahrzehnten zeichnet sich eine immer weiter voranschreitende  Digitalisierung der Gesellschaft ab. Manche Wirtschaftsbetriebe und Einrichtungen haben sich schon früh damit arrangiert und entsprechende Produkte angeschafft. Dies umfasst fertig käufliche Hard- und Softwareprodukte, sowie individuell entwickelte Lösungen. Diese sind, wie die meisten digitalen Güter, einem schnellen Alterungsprozess unterworfen. Das können Schnittstellen sein, welche vom Hersteller nicht mehr unterstützt werden, oder eine neue Betriebssystem Version die nicht mehr Unterstützt wird. Wenn die Kunden direkt mit einer Software interagieren, ist Userexperience und Design ein nicht zu unterschätzender Faktor.

Die meisten Softwareinfrastrukturen wachsen meist mit einem Unternehmen und seinem Bedarf nach digitalen Lösungen. Das bedeutet, dass eine verworrene Struktur aus Abhängigkeiten entstehen kann. Das ist dabei eher die Regel als die Ausnahme.

\ac{EDV}-Systeme haben nicht die klassischen Verschleißerscheinungen, so wie man sie von klassischen Betriebsmitteln kennt. Allerdings entsteht auch so ein Interesse nach einem gewissen Nutzungszeitraum die bestehende Software zu ersetzen. Man bezeichnet diesen Alterungsprozess, welchen man eingehend in der Softwarequalitätsforschung untersucht, als Softwarealterung. Software ist \glqq weich\grqq{} und man sollte annehmen sie sei leicht änderbar und wartbar. Dies kann allerdings mit fortschreitendem Alter teurer sein als die Migration zu einem neuen System. Firmen die nicht mit der Zeit gehen werden schnell als alt und uninnovativ wahrgenommen. Dies kann sich schnell auf den Umsatz eines Unternehmens auswirken. Deshalb sind Migrationen gerade im Bereich des Web-Developments besonders häufig, bei denen man von Grund auf ein neues System erstellt.\cite{Wagner2014}

Es gibt mehrere Fälle von Softwaremigrationen. In manchen Fällen kann Software Hardware ersetzen. In andere Fällen ersetzt wiederum Software Hardware. Wesentlich häufiger ist jedoch das alte Software durch neue ersetzt wird, sowie alte Hardware durch ihre neueren Pendants. 

\section{Vorstellung des Unternehmens}
Der Aufgabestellung wurde von der Firma "SitePoint" festgelegt. Das Unternehmen spezialisiert in den Bereichen Content Management System, Mobile Web Applications und e-commerce. Die Mitarbeiter sind .Net-Experten, die hochwertige Software auf Microsoft-Technologie setzen. "SitePoint" ist das einzige Unternehmen im Saarland, das Umbraco CMS verwendet. M. Sc. Thomas Beckert ist den Geschäftsführer der Firma und auch den Aufgabegeber der gegenwärtigen Arbeit. \cite{SitePoint2018}  

\section{Motivation}

In dieser Abschlussarbeit wird die Umstellung eines Bestellsystems eines Catering Services abgebildet werden. Dies bedeutet, das es sich bei dem Thema der Arbeit um eine Software zu Software Migration handelt. Genauer um eine modellgetriebene Softwaremigration.

Dabei wird das aus dem beginn dieses Jahrtausend stammende Bestellsystem, welches auf einem \ac{WIC}-Plugin basiert und einem \ac{ASP}-Backend. Da die damals verwendeten Technologien vom Hersteller Microsoft seit geraumer Zeit End-of-Life gesetzt wurden, ist eine Migration zu einem aktuelleren Technologiestack zwingend erforderlich. Das Ziel des Migrationsprojekt ist es, das aktuelle Bestellsystem, welches immernoch auf ASP basiert, durch einen modernen Technologiestack zu ersetzen. 

Dazu zählt die neue Kozeptionierung des Frontend. Dies soll zur Verbesserung und Erleichterung der Bedienbarkeit führen. Dazu wird eine neue Seite, welche auf Umbraco und ASP.NET basiert, erstellt.

\section{Zielsetzung} 

Dieses Projekt wird in zwei Paketen aufgeteilt. Das Ziel des ersten Pakets A ist dem Auftraggeber verschiedene Möglichkeiten zu bieten, die Inhalte seiner Seite zu editieren und zu ändern.

Im Paket B werden angefordert, dass ein Bestellsystem die Kommunikation zwischen dem Auftraggeber und seinem Kunde verwaltet. Weiterhin soll das System Buchhaltung und Auftragsspeicherung steuern. Zusätzlich wird die Erarbeitung eines Konzepts angefordert, das die Daten von der veralteten, webbasierten Software zu den obengenannten Anforderungen übertragen müssen. 

\section{Aufgabenstellung}

Im nachfolgenden Kapitel werden alle verwendeten Technologien erörtert und kurz erklärt. Danach wird im darauf folgenden Kapitel der technische und optische Stand der aktuellen Internetpräsenz analysiert.

 Im darauf folgenden vierten Kapitel erfolgt die Erfassung und Anforderungsanalyse der Problemstellung deren theoretischen Lösungen. Im nachfolgenden Kapitel wird darauf aufgebaut und es erfolgt die praktische Umsetzung der Lösungssätze.
Im letzten Kapitel wird die Qualität der Umsetzung erörtert und persönliche Designentscheidungen begründet.  Danach folgt ein Ausblick auf weitere Verbesserungsmöglichkeiten.


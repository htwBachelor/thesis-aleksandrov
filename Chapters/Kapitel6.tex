\chapter{Schluss}
\section{Zusammenfassung}

Ziel dieser Arbeit war es, eine veraltete webbasierte Software zu analysieren, mit der eine Migration zu einem neuen System durchgeführt werden sollte. Hierbei lag das Augenmerk auf der Problemerkennung und -darstellung anhand der Analyse und der Anforderungen sowie auf der Problemlösung.
Dazu wurde zunächst die allgemeine Aufgabenstellung definiert und eingegrenzt. Die erstellten Anwendungsfälle haben einen Überblick über die zu bearbeitenden Aufgaben gegeben. Eine IST-Analyse wurde durchgeführt, um in einer Detailansicht jeden Punkt der Software zu betrachten. Danach wurden Werkzeuge angewendet, um eine Bewertung dieser Punkte durchzuführen. Es wurde festgestellt, welche Anforderungsmerkmale für das neue Websystem gelten und welche alten Funktionalitäten beibehalten werden sollten. Als Basis für die neue Software wurde Umbraco CMS ausgewählt. Die neue erstellten Datenbanken mussten an die alten angepasst werden, damit eine reibungslose Migration durchgeführt werden konnte.

Im Konzept wurden die allgemeinen Anforderungsmerkmale konkretisiert. Ein Entwurf der fachlichen und technischen Architektur wurde für jeden zugehörigen Teil der vorgegebenen Anforderungen festgelegt, ebenso wie die Implementierung. Die Daten der Kunden wurden erfolgreich in der Member-Section von Umbraco gespeichert, womit der Auftraggeber die Kundendaten editieren oder löschen kann. Auch die Möglichkeit für den Auftraggeber, private Information an die Kunden zu senden, wurde gegeben. Das Konzept zur Migration wurde theoretisch erstellt. Der Prozess erfolgte durch Konvertierung der Datenbankdateien in eine CSV-Datei und anhand des SurfaceController wurden die Dateien in die neue Datenbank übertragen. Weitere Tests der Software und Verbesserungen hinsichtlich der Kommunikation, der Datenbankverbindungen und der Artikelverwaltung sind zu empfehlen. Bei der Bewertung der Implementierung hat sich gezeigt, dass die wichtigsten funktionalen Anforderungen erfüllt sind und sich das System leicht durch zusätzliche Funktionen erweitern lässt.


\section{Ausblick und Kritik}

Der Funktionsumfang der entwickelten Software bietet dem Auftraggeber eine flexible und unkomplizierte Verwaltungsmöglichkeit.
Einige Merkmale, die aus den Anforderungen ermittelt wurden, wie die Verbindung von Datenbank zum Frontend, fehlen teilweise noch. Eine Implementierung dieser Komponenten kann vorgenommen werden, da diese leicht ins Konzept zu integrieren sind. Zu diesen Merkmalen zählen auch die Verbindungen zur noch nicht entwickelten E-Mail- und Umsatzverwaltung. Das Kommunikationskonzept wurde erfolgreich entwickelt, aber aufgrund einer  Erwartung, dass die Kommunikation im Member-Bereich sich befinden musste, wurde das Konzept zurückgewiesen. Aufgrund von fehlenden verbundenen Datenbanken funktionieren die Artikelverwaltung, die Übersicht und die Detailansicht nicht. Eine Erläuterung der E-Mail- und der Umsatzverwaltung konnte aus Zeitgründen nicht erfolgen.



\chapter{Zusammenfassung und Ausblick}

In dieser Arbeit wurde eine veraltete webbasierte Bestellsystem. Nach der Analyse wurde festgestellt, dass eine neues Konzept aufgebaut werden muss, damit eine Migration der Software geschehen werden kann. Die Anforderungen wurden festgelegt und wurde Ubraco CMS als Migrationsziel entschieden. Nach tiefe Recherche, Forschungen und Tests wurden die Konzeption und Implementierung erstellt. Während Arbeitsverlauf wurden viele Quellen genutzt. Das Konzept wurde Teilweise beendet. Zwei Unterkapitel wurden nicht erläutert, wegen weniger Information und Zeitaufwand. Vorteile nach der Migration wurde erreicht. Der Auftraggeber hat verbreitete Möglichkeit und Flexibilität für die Verwaltung des Frontends. Kundenerfassung wurde erfolgreich bestanden. Es wurde ein Missverständnis zu Kommunikation erschient. Deswegen wurde die Kommunikationskonzept zurückgewiesen. Die Daten des Kunden wurden erfolgreich im Member-Section von Umbraco gespeichert. Dort kann der Auftraggeber die Kundendaten editieren oder löschen. Die Möglichkeit, die dem Auftraggeber ergibt, private Information zum Kunde zu senden, wurde eben bestanden. Aufgrund nicht verbundenen Datenbanken funktioniert die Artikelverwaltung, Übersicht und Detailansicht nicht.
Das Konzept zur Migration wurde theoretisch erstellt. Das Prozess wird durch Konvertierung der Datenbankdateien auf CSV-Datei realisiert und anhand SurfaceControllers in der neuen Datenbank übertragen. Es wird empfohlen weitere Testen der Software und Verbesserungen für Kommunikation, Datenbankverbindungen und Artikelverwaltung.
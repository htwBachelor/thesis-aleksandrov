\chapter{Schluss}
\section{Zusammenfassung}

Ziel dieser Arbeit war es, eine veraltete webbasierte Software zu analysieren, welche eine Migration zu ein neuen System durchgeführt werden musste. Hierbei lag der Augenmerk auf der Problemerkennung und -anzeige durch die Analyse und Anforderungen, sowie die Problemlösung.

Dazu werde zunächst die allgemeine Aufgabenstellung definiert und eingegrenzt. Erststelle wurden die Webtechnologie und Werkzeuge vorgestellt. 
Eine IST-Analyse wurde durchgeführt, um in einer Detailansicht jedes Punkt der Software betrachtet zu werden. Danach werden Werkzeuge verwenden, um eine Bewertung dieser Punkte zu ergeben.
Die Anforderungsmerkmale für das neuen Websystem wurden festgestellt, sowie Behalten alter Funktionalitäten. Umbraco CMS wurde als Basis neuer Software entschieden. Neue erstellte Datenbanken mussten zu den alten angepasst werden, damit eine reibungslose Migration durchgeführt werden.

Im Konzept wurden die allgemeinen Anforderungsmerkmale konkretisiert.
Ein Entwurf mit fachlicher und technischer Architektur wurde zu jedem zugehörigen Teil der vorgegebenen Anforderungen festgelegt, sowie die Implementierung. 
Die Daten des Kunden wurden erfolgreich im Member-Section von Umbraco gespeichert. Dort kann der Auftraggeber die Kundendaten editieren oder löschen. Die Möglichkeit, die der Auftraggeber hat, private Information zum Kunde zu senden, wurde eben bestanden.
Das Konzept zur Migration wurde theoretisch erstellt. Das Prozess wird durch Konvertierung der Datenbankdateien auf CSV-Datei realisiert und anhand SurfaceControllers in der neuen Datenbank übertragen. Es wird empfohlen weitere Testen der Software und Verbesserungen für Kommunikation, Datenbankverbindungen und Artikelverwaltung.  
Bei der Bewertung der Implementierung wurden hat sich gezeigt, dass ein Großteil der Anforderungen umgesetzt wurde. Es erfüllt die wichtigsten funktionalen Anforderungen und lässt sich leicht durch zusätzliche Funktionen erweitern.

Im Ganzen lässt sich sagen, dass das implementierte Systeme für den geforderten Migration theoretisch zu sehen ist.

\section{Ausblick und Kritik}

Der Funktionsumfang der entwickelten Software bietet dem Auftraggeber eine flexible und unkomplizierte Verwaltungsmöglichkeit. 

Einige Merkmale, die aus Anforderungen ermittelt wurden, wie die Verbindung von Datenbank zum Frontend, fehlen teilweise noch. Eine Implementierung dieser Komponenten kann vorgenommen werden. da diese leicht ins Konzept zu integrieren sind. Zu diesen Merkmalen zählen neben die Verbindungen von noch nicht entwickelten E-Mail- und Umsatzverwaltung. 
Die Kommunikationskonzept wurde erfolgreich entwickelt, aber wegen Missverständnis wurde es zurückgewiesen.
Aufgrund von fehlenden verbundenen Datenbanken funktionieren die Artikelverwaltung, Übersicht und Detailansicht nicht.
Die Zeit war nicht für E-Mail- und Umsatzverwaltung nicht genug, deswegen sie wurden nicht erläutern. 
